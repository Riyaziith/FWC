\documentclass{article}
\usepackage{graphicx}
\usepackage{amsmath}
\usepackage{commath}
\usepackage{gensymb}
\usepackage{enumitem}
\usepackage{multicol}
\usepackage[margin=0.5in]{geometry}

\newlist{choices}{enumerate}{1}
\setlist[choices]{label*=(\Alph*)}
\newcommand{\choice}{\item}

\SetEnumitemKey{twocol}{
	before=\raggedcolumns\begin{multicols}{2},
	after=\end{multicols}}
\SetEnumitemKey{threecol}{
	before=\raggedcolumns\begin{multicols}{3},
	after=\end{multicols}}
\SetEnumitemKey{fourcol}{
	before=\raggedcolumns\begin{multicols}{4},
	after=\end{multicols}}	
\begin{document}
\textbf{}
\\
\\Latex Math Assignment ncert-exemplar/12/leep210.pdf- Excercise 10.3\\
\textbf{}
\\
\textbf{Short Answer (S.A)}
`:\\
1. Find the unit vector in the direction of sum of vectors $\vec{a}$= $2\hat{i}-\hat{j}+\hat{k}$ $\text{ and }$ $\vec{b}=2\hat{j}+\hat{k}$.
\\
\\
2. If $\vec{a}$=$\hat{i}+\hat{j}+2\hat{k}$ $\text{ and }$ $\vec{b}$=$2\hat{i}+\hat{j}-2\hat{k}$, find the unit vector in the direction of
\\
(i) 6$\vec{a}$   $\hspace{2cm}$ (ii)2$\vec{a}$-$\vec{b}$
\\
\\
3. Find a unit vector in the direction of $\overline{PQ} $, where P and Q have co-ordinates(5,0,8) and (3,3,2),respectively.
\\
\\
4. If $\vec{a}$ $\text{and}$ $\vec{b}$ are the postion vectors of A and B, respectively, find the position vector of a point C in BA produced such that BC=1.5BA.
\\
\\
5. Using vectors, find the value of $k$ such that the points $(k,-10,3)$, $(1,-1,3)$ $\text{ and }$ $(3,5,3)$ are collinear.
\\
\\
6. A vector $\vec{r}$ is inclined at equal angles to the three axis. If the magnitude of $\vec{r}$ is $2\sqrt{3}$ units, find $\vec{r}$.
\\
\\
7. A vector $\vec{r}$ has a magnitude 14 and direction ratios 2,3,-6. Find the direction cosines and components of $\vec{r}$, given that $\vec{r}$ makes an acute angle with x-axis.
\\
\\
8. Find a vector of magnitude 6, which is perpendicular to both the vectors $2\hat{i}-\hat{j}$+$2\hat{k}$ $\text{ and }$ $4\hat{i}-\hat{j}+3\hat{k}$.
\\
\\
9. Find the angle between the vectors $2\hat{i}-\hat{j}+\hat{k}$ $\text{and}$ $3\hat{i}+4\hat{j}-\hat{k}$.
\\
\\
10. If $\vec{a}+\vec{b}+\vec{c}$=0, show that $\vec{a}\times\vec{b}$=$\vec{b}\times\vec{c}$=$\vec{c}\times\vec{a}$. Interpret the result geometrically?
\\
\\
11. Find the sine of the angle between the vectors $\vec{a}=3\vec{i}+\vec{j}+2\vec{k}$ $\text{ and }$ $\vec{b}=2\vec{i}-2\vec{j}+4\vec{k}$.
\\
\\
12. If A,B,C,D  are the points with position vectors $\hat{i}+\hat{j}-\hat{k}$, $2\hat{i}-\hat{j}+3\hat{k}$, $2\hat{i}-3\hat{k}$, $3\hat{i}$-$2\hat{j}$+$\hat{k}$, respectively, find the projection of $\overline{AB}$ $\text{ along }$ $\overline{CD}$.
\\
\\
13. Using vectors, find the area of $\triangle{ABC}$ with vertices A(1,2,3), B(2,-1,4) and C(4,5,-1).
\\
\\
14. Using vectors, prove that the parallelogram on the same base and between the same parallels are equal in area.
\\
\\
\textbf{Long Answer(L.A)}
15. Prove that in any $\triangle{ABC}$, cos A=$\dfrac{b^2+c^2-a^2}{2bc}$, where a,b,c are the magnitudes of the sides opposite to the vertices A,B,C respectively.
\\
\\
16. If $\vec{a}$, $\vec{b}$, $\vec{c}$ ,determine the vertices of a triangle, show that $\dfrac{1}{2}$ $\left[\vec{b} \times\vec{c}+\vec{c} \times\vec{a}+\vec{a}\times\vec{b} \right]$ gives the vector area of the trianlge. Hence deduce the condition that the three points $\vec{a},\vec{b},\vec{c},$ are collinear. Also find the unit vector normal to the plane of the triangle.
\\
\\
17. Show that area of the parallelogram whose diagonals are given by $\vec{a}\times\vec{b}$ $\text{is}$ $\dfrac{|\vec{a}\times\vec{b}|}{2}$. Also find the area of the parallelogram whose diagonals are $2\hat{i}-\hat{j}+\hat{k}$ $\text{and}$ $\hat{i}+3\hat{j}-\hat{k}$.
\\
\\
18. If $\vec{a}$ = $\hat{i}+\hat{j}+\hat{k}$ $\text{and}$ $\vec{b}$ = $\hat{j}-\hat{k}$, find a the vector $\vec{c}$ such that $\vec{a}\times\vec{c}$ = $\vec{b}$ $\text{ and}$ $\vec{a}$.$\vec{c}$ = 3.
\\
\\
\textbf{Objective Type Questions}
Choose the correct answer from the given four options in each of the Excercise from 19 to 33(M.C.Q)
\\
\\
19. The vector in the direction of the vector $\hat{i}-2\hat{j}+2\hat{k}$ that has magnitude 9 is
\begin {choices}[twocol]
\choice $\hat{i}-2\hat{j}+2\hat{k}$
\choice $\hat{i}-2\hat{j}$
\choice $3(\hat{i}-2\hat{j}+2\hat{k})$
\choice $9(\hat{i}-2\hat{j}+2\hat{k})$
\end{choices}
\\
\\
20. The position vector of the point which divides the join of points 2$\vec{a}$-3$\vec{b}$ $\text{and}$ $\vec{a}+\vec{b}$ in the ratio 3:1 is
\begin {choices}[fourcol]
\choice $\dfrac{3\vec{a}-2\vec{b}}{2}$
\choice $\dfrac{7\vec{a}-8\vec{b}}{4}$
\choice $\dfrac{\vec{3a}}{4}$
\choice $\dfrac{\vec{5a}}{4}$
\end{choices}
\\
\\
21. The vector having intial and terminal points as (2,5,0)and (-3,7,4),respectively is
\begin {choices}[twocol]
\choice -$\hat{i}+12\hat{j}+4\hat{k}$
\choice $5\hat{i}+2\hat{j}-4\hat{k}$
\choice $5\hat{i}+2\hat{j}+4\hat{k}$
\choice $\hat{i}+\hat{j}+\hat{k}$
\end{choices}
\\
\\
22. The angles between two vectors $\vec{a}$ $\text{and}$ $\vec{b}$ with magnitude $\sqrt{3}$ $\text{ and }$ 4, respectively, and $\vec{a}$, $\vec{b}$= $2\sqrt{3}$ is
\begin {choices}[fourcol]
\choice $\dfrac{\pi}{6}$
\choice $\dfrac{\pi}{3}$
\choice $\dfrac{\pi}{2}$ 
\choice $\dfrac{5\pi}{2}$
\end{choices}
\\
\\
23. Find the value of $\lambda$ such that the vectors $\vec{a}=2\hat{i}+\lambda\hat{j}+\hat{k}$ $\text{and}$ $\vec{b}=\hat{i}+2\hat{j}+3\hat{k}$ are orthogonal.
\begin {choices}[fourcol]
\choice 0
\choice 1 
\choice $\dfrac{3}{2}$
\choice $-\drac{5}{2}$
\end{choices}
\\
\\
24. The value of $\lambda$ for which the vectors $3\hat{i}-6\hat{j}+\hat{k}$ $\text{and}$,  $2\hat{i}-4\hat{j}$+$\lambda\hat{k}$ are parallel is
\begin {choices}[fourcol]
\choice $\dfrac{2}{3}$
\choice $\dfrac{3}{2}$
\choice $\dfract{5}{2}$
\choice $\dfract{2}{5}$
\end{choices}	
\\
\\
25. The vector from origin to the points A and B are $\vec{a} = 2\hat{i}-3\hat{j}+2\hat{k}$ $\text{and}$  $\vec{b}$ = $2\hat{i}+3\hat{j}+\hat{k}$, respectively, then the area of $\triangle {OAB}$ is
\begin {choices}[fourcol]
\choice 340 
\choice $\sqrt{25}$
\choice $\sqrt{229}$
\choice $\dfract{1}{2}\sqrt{229}
\end{choices}
\\
\\
26. For any vector $\vec{a}$, the value of $(\vec{a}\times\hat{i})^2+(\vec{a}\times\hat{j})^2 + (\vec{a}\times\hat{k})^2$is equal to 
\begin {choices}[fourcol]
\choice a 
\choice 3a
\choice 4a
\choice 2a
\end{choices}
\\
\\
27. If $|\vec{a}|=10$, $|\vec{b}|=2$ $\text{ and }$  $\vec{a}$, $\vec{b}$=12, then value of $|\vec{a}\times\ \vec{b}|$ is
\begin {choices}[fourcol]
\choice 5 
\choice 10 
\choice 14 
\choice 16
\end{choices}
\\
\\
28. The vectors $\lambda\hat{i}+\lambda\hat{j}+2\hat{k}$, $\hat{i}+\lambda\hat{j}-\hat{k}$ $\text{ and }$ $2\hat{i}-\hat{j}+\lambda{k}$ are coplanar if
\begin {choices}[fourcol]
\choice	$\lambda=-2$
\choice $\lambda=0$
\choice $\lambda=1$
\choice	$\lambda=-1$
\end{choices}
\\
\\
29. If $\vec{a}$, $\vec{b}$, $\vec{c}$ are unit vectors such that $\vec{a}$+$\vec{b}$+$\vec{c}$=0, then the value of $\vec{a}.\vec{b}+\vec{b}.\vec{c}+\vec{c}.\vec{a}$ is
\begin {choices}[fourcol]
\choice 1
\choice 3
\choice $\dfrac{-3}{2}$
\choice None of these
\end{choices}
\\
\\
30.  Projection vector of $\vec{a}$ on $\vec{b}$ is
\begin {choices}[fourcol]
\choice $\left(\dfrac{\vec{a}.\vec{b}}{|\vec{b}|^2}\right)$
\choice $\dfrac{\vec{a}.\vec{b}}{|\vec{b}|}$
\choice $\dfrac{\vec{a}.\vec{b}}{|\vec{a}|}$
\choice $\left(\dfrac{\vec{a}.\vec{b}}{|\vec{a}|^2}\right)$
\end{choices}
\\
\\
31. If $\vec{a},\vec{b},\vec{c}$ are the three vectors such that $\vec{a}+\vec{b}+\vec{c}=0$ $\text{ and }$ $|\vec{a}|=2$, $|\vec{b}|$=3, $|\vec{c}|$=5, the value of $\vec{a}.\vec{b}+\vec{b}.\vec{c}+\vec{c}.\vec{a}$ is
\begin {choices}[fourcol]
\choice 0
\choice 1	
\choice -19
\choice 38
\end{choices}
\\
\\
32. If $|\vec{a}|=4$ $\text{and}$  $-3\leq\lambda\leq2$, then the range of $|\lambda\vec{a}|$ is
\begin {choices}[fourcol]
\choice $\left[0,8\right]$
\choice $\left[-12,8\right]$
\choice $\left[0,12\right]$	
\choice $\left[8,12\right]$
\end{choices}
\\
\\
33. The number of vectors of unit length perpendicular to the vectors $\vec{a}=2\hat{i}+\hat{j}+2\hat{k}$ $\text{ and }$ $\vec{b}=\hat{j}+\hat{k}$ is
\begin {choices}[fourcol]
\choice  one
\choice  two
\choice three
\choice infinite
\end{choices}
\\
\\
Fill in the blanks in each of the Excercises from 34 to 40.
\\
\\
34. The vector $\vec{a}+\vec{b}$ bisects the angle between the non-collinear vectors $\vec{a}$ $\text{ and }$ $\vec{b}$ if \rule{1cm}{0.15mm}
\\
\\
35. If $\vec{r}.\vec{a}=0$, $\vec{r}.\vec{b}=0$ $\text{and}$ $\vec{r}.\vec{c}=0$ for some non-zero vector $\vec{r}$, then the value of $\vec{a}.(\vec{b}$ $\times$ $\vec{c})$ is \rule{1cm}{0.15mm}
\\
\\
36. The vectors $\vec{a}=3\hat{i}-2\hat{j}+2\hat{k}$ $\text{ and }$ $\vec{b}=\hat{i}-2\hat{k}$ are the adjancent sides of a parallelogram. The acute angle between its diagonals is \rule{1cm}{0.15mm}
\\
\\
37. The values of $k$ for which $|k\vec{a}|$ $<$ $|\vec{a}|$ $\text{and}$ $k\vec{a}$+$\dfrac{1}{2}$ $\vec{a}$ is parallel to $\vec{a}$ holds true are \rule{1cm}{0.15mm}
\\
\\
38. The value of the expression $|\vec{a}\times\vec{b}|$+ $(\vec{a}.\vec{b})$ is \rule{1cm}{0.15mm}
\\
\\
39. If $|\vec{a}\times\vec{b}|$ + $|\vec{a}.\vec{b}|$=144 $\text{and}$  $|\vec{a}|$=4, then $|\vec{b}|$ is equal to \rule{1cm}{0.15mm}
\\
\\
40. If $\vec{a}$ is  any non-zero vector, then $(\vec{a}.\hat{i})\hat{i}$+$(\vec{a}.\hat{j})\hat{j}$+$(\vec{a}.\hat{k})$ $\hat{k}$ equals \rule{1cm}{0.15mm}
\\
\\
State \textbf{True} or \textbf{False} in each of the following Exercises.
\\
\\
41. If $|\vec{a}|=|\vec{b}|$, then necessarily it implies $\vec{a}=\pm\vec{b}$.
\\
\\
42. Position vector of point P is a vector whose intial point is origin.
\\
\\
43. If $|\vec{a}+\vec{b}$=$|\vec{a}-\vec{b}|$, then the vectors $\vec{a}$ $\text {and}$ $\vec{b}$ are orthogonal.
\\
\\
44. The formula $(\vec{a}+\vec{b})$ = $\vec{a}$+$\vec{b}$ + 2$\vec{a}\times\vec{b}$ is valid for non-zero vectors $\vec{a}$ $\text{and}$ $\vec{b}$.
\\
\\
45. If $\vec{a}$ $\text{ and }$ $\vec{b}$ are adjacent sides of a rhombus, then $\vec{a}.\vec{b}$.=0.

\end{document}
