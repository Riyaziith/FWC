\documentclass{article}
\usepackage{graphicx}
\usepackage{amsmath}
\usepackage{commath}
\usepackage{gensymb}
\usepackage{enumitem}
\usepackage{multicol}
\usepackage{ragged2e}
\usepackage[margin=0.5in]{geometry}
\providecommand{\abs}[1]{\left\vert#1\right\vert}

	
\begin{document}
Latex Math Assignment ncert-exemplar/12/leep210.pdf- Excercise 10.3\\
\textbf{Short Answer (S.A):}

\begin{enumerate}
\item Find the unit vector in the direction of sum of vectors $\vec{a}$= $2\hat{i}-\hat{j}+\hat{k}$ $\text{ and }$ $\vec{b}=2\hat{j}+\hat{k}$.

\item If $\vec{a}$=$\hat{i}+\hat{j}+2\hat{k}$ $\text{ and }$ $\vec{b}$=$2\hat{i}+\hat{j}-2\hat{k}$, find the unit vector in the direction of
	\begin{enumerate}
		\item 6$\vec{a}$   
		\item 2$\vec{a}$-$\vec{b}$
	\end{enumerate}

\item Find a unit vector in the direction of $\overline{PQ} $, where P and Q have co-ordinates(5,0,8) and (3,3,2),respectively.

\item If $\vec{a}$ $\text{and}$ $\vec{b}$ are the postion vectors of A and B, respectively, find the position vector of a point C in BA produced such that BC=1.5BA.

\item Using vectors, find the value of $k$ such that the points $(k,-10,3)$, $(1,-1,3)$ $\text{ and }$ $(3,5,3)$ are collinear.

\item A vector $\vec{r}$ is inclined at equal angles to the three axis. If the magnitude of $\vec{r}$ is $2\sqrt{3}$ units, find $\vec{r}$.


\item A vector $\vec{r}$ has a magnitude 14 and direction ratios 2,3,-6. Find the direction cosines and components of $\vec{r}$, given that $\vec{r}$ makes an acute angle with x-axis.


\item Find a vector of magnitude 6, which is perpendicular to both the vectors $2\hat{i}-\hat{j}$+$2\hat{k}$ $\text{ and }$ $4\hat{i}-\hat{j}+3\hat{k}$.


\item Find the angle between the vectors $2\hat{i}-\hat{j}+\hat{k}$ $\text{and}$ $3\hat{i}+4\hat{j}-\hat{k}$.


\item If $\vec{a}+\vec{b}+\vec{c}$=0, show that $\vec{a}\times\vec{b}$=$\vec{b}\times\vec{c}$=$\vec{c}\times\vec{a}$. Interpret the result geometrically?



\item Find the sine of the angle between the vectors $\vec{a}=3\hat{i}+\hat{j}+2\hat{k}$ $\text{ and }$ $\vec{b}=2\hat{i}-2\hat{j}+4\hat{k}$.


\item If A,B,C,D  are the points with position vectors $\hat{i}+\hat{j}-\hat{k}$, $2\hat{i}-\hat{j}+3\hat{k}$, $2\hat{i}-3\hat{k}$, $3\hat{i}$-$2\hat{j}$+$\hat{k}$, respectively, find the projection of $\overline{AB}$ $\text{ along }$ $\overline{CD}$.


\item Using vectors, find the area of $\triangle{ABC}$ with vertices A(1,2,3), B(2,-1,4) and C(4,5,-1).


\item Using vectors, prove that the parallelogram on the same base and between the same parallels are equal in area.
\end{enumerate}


\textbf{Long Answer(L.A)}
\begin{enumerate}[resume]
\item Prove that in any $\triangle{ABC}$, cos A=$\frac{b^2+c^2-a^2}{2bc}$, where a,b,c are the magnitudes of the sides opposite to the vertices A,B,C respectively.


\item If $\vec{a}$, $\vec{b}$, $\vec{c}$ ,determine the vertices of a triangle, show that $\frac{1}{2}$ $\left[\vec{b} \times\vec{c}+\vec{c} \times\vec{a}+\vec{a}\times\vec{b} \right]$ gives the vector area of the trianlge. Hence deduce the condition that the three points $\vec{a},\vec{b},\vec{c},$ are collinear. Also find the unit vector normal to the plane of the triangle.

\item Show that area of the parallelogram whose diagonals are given by $\vec{a}\times\vec{b}$ is $\frac{\abs{\vec{a}\times\vec{b}}}{2}$. Also find the area of the parallelogram whose diagonals are $2\hat{i}-\hat{j}+\hat{k}$ $\text{and}$ $\hat{i}+3\hat{j}-\hat{k}$.


\item If $\vec{a}$ = $\hat{i}+\hat{j}+\hat{k}$ $\text{and}$ $\vec{b}$ = $\hat{j}-\hat{k}$, find a the vector $\vec{c}$ such that $\vec{a}\times\vec{c}$ = $\vec{b}$ $\text{ and}$ $\vec{a}$.$\vec{c}$ = 3.

\end{enumerate}
\textbf{Objective Type Questions}
\\
Choose the correct answer from the given four options in each of the Excercise from 19 to 33(M.C.Q)
\begin{enumerate}[resume]
\item The vector in the direction of the vector $\hat{i}-2\hat{j}+2\hat{k}$ that has magnitude 9 is
	\begin{enumerate}
\item $\hat{i}-2\hat{j}+2\hat{k}$
\item $\hat{i}-2\hat{j}$
\item $3(\hat{i}-2\hat{j}+2\hat{k})$
\item $9(\hat{i}-2\hat{j}+2\hat{k})$
\end{enumerate}


\item The position vector of the point which divides the join of points 2$\vec{a}$-3$\vec{b}$ $\text{and}$ $\vec{a}+\vec{b}$ in the ratio 3:1 is
	\begin{enumerate}
\item $\frac{3\vec{a}-2\vec{b}}{2}$
\item $\frac{7\vec{a}-8\vec{b}}{4}$
\item $\frac{\vec{3a}}{4}$
\item $\frac{\vec{5a}}{4}$
\end{enumerate}



\item The vector having intial and terminal points as (2,5,0)and (-3,7,4),respectively is
	\begin{enumerate}
\item -$\hat{i}+12\hat{j}+4\hat{k}$
\item $5\hat{i}+2\hat{j}-4\hat{k}$
\item $5\hat{i}+2\hat{j}+4\hat{k}$
\item $\hat{i}+\hat{j}+\hat{k}$
\end{enumerate}


\item The angles between two vectors $\vec{a}$ $\text{and}$ $\vec{b}$ with magnitude $\sqrt{3}$ $\text{ and }$ 4, respectively, and $\vec{a}$, $\vec{b}$= $2\sqrt{3}$ is
	\begin{enumerate}
\item $\frac{\pi}{6}$
\item $\frac{\pi}{3}$
\item $\frac{\pi}{2}$ 
\item $\frac{5\pi}{2}$
\end{enumerate}

\item Find the value of $\lambda$ such that the vectors $\vec{a}=2\hat{i}+\lambda\hat{j}+\hat{k}$ $\text{and}$ $\vec{b}=\hat{i}+2\hat{j}+3\hat{k}$ are orthogonal.
	\begin{enumerate}
\item 0
\item 1 
\item $\frac{3}{2}$
\item $-\frac{5}{2}$
	\end{enumerate}

\item The value of $\lambda$ for which the vectors $3\hat{i}-6\hat{j}+\hat{k}$ $\text{and}$,  $2\hat{i}-4\hat{j}$+$\lambda\hat{k}$ are parallel is
	\begin{enumerate}
\item $\frac{2}{3}$
\item $\frac{3}{2}$
\item $\frac{5}{2}$
\item $\frac{2}{5}$
	\end{enumerate}	

\item The vector from origin to the points A and B are $\vec{a}$ = $2\hat{i}-3\hat{j}+2\hat{k}$ $\text{and}$  $\vec{b}$ = $2\hat{i}+3\hat{j}+\hat{k}$, respectively, then the area of $\triangle {OAB}$ is
	\begin{enumerate}
\item 340 
\item $\sqrt{25}$
\item $\sqrt{229}$
\item $\frac{1}{2}\sqrt{229}$
\end{enumerate}


\item For any vector $\vec{a}$, the value of $(\vec{a}\times\hat{i})^2+(\vec{a}\times\hat{j})^2 + (\vec{a}\times\hat{k})^2$is equal to 
	\begin{enumerate}
\item a 
\item 3a
\item 4a
\item 2a
\end{enumerate}


\item If $\abs{\vec{a}}$=10, $\abs{\vec{b}}=2$ $\text{ and }$  $\vec{a}$, $\vec{b}$=12, then value of $\abs{\vec{a}\times\vec{b}}$ is
	\begin{enumerate}
\item 5 
\item 10 
\item 14 
\item 16
\end{enumerate}


\item The vectors $\lambda\hat{i}+\lambda\hat{j}+2\hat{k}$, $\hat{i}+\lambda\hat{j}-\hat{k}$ $\text{ and }$ $2\hat{i}-\hat{j}+\lambda\hat{k}$ are coplanar if
	\begin{enumerate}
\item	$\lambda=-2$
\item $\lambda=0$
\item $\lambda=1$
\item	$\lambda=-1$
\end{enumerate}


\item If $\vec{a}$, $\vec{b}$, $\vec{c}$ are unit vectors such that $\vec{a}$+$\vec{b}$+$\vec{c}$=0, then the value of $\vec{a}.\vec{b}+\vec{b}.\vec{c}+\vec{c}.\vec{a}$ is
	\begin{enumerate}
\item 1
\item 3
\item $\frac{-3}{2}$
\item None of these
\end{enumerate}



\item Projection vector of $\vec{a}$ on $\vec{b}$ is
	\begin{enumerate}
\item $\left(\dfrac{\vec{a}.\vec{b}}{\abs{\vec{b}}^2}\right)$
\item $\frac{\vec{a}.\vec{b}}{\abs{\vec{b}}}$
\item $\frac{\vec{a}.\vec{b}}{\abs{\vec{a}}}$
\item $\left(\dfrac{\vec{a}.\vec{b}}{\abs{\vec{a}}^2}\right)$
\end{enumerate}



\item If $\vec{a},\vec{b},\vec{c}$ are the three vectors such that $\vec{a}+\vec{b}+\vec{c}=0$ $\text{ and }$ $|\vec{a}|=2$, $|\vec{b}|$=3, $|\vec{c}|$=5, the value of $\vec{a}.\vec{b}+\vec{b}.\vec{c}+\vec{c}.\vec{a}$ is
	\begin{enumerate}
\item 0
\item 1	
\item -19
\item 38
\end{enumerate}



\item If $|\vec{a}|=4$ $\text{and}$  $-3\leq\lambda\leq2$, then the range of $|\lambda\vec{a}|$ is
	\begin{enumerate}
\item $\left[0,8\right]$
\item $\left[-12,8\right]$
\item $\left[0,12\right]$	
\item $\left[8,12\right]$
\end{enumerate}

\item The number of vectors of unit length perpendicular to the vectors $\vec{a}=2\hat{i}+\hat{j}+2\hat{k}$ $\text{ and }$ $\vec{b}=\hat{j}+\hat{k}$ is
	\begin{enumerate}
\item one
\item  two
\item three
\item infinite
\end{enumerate}
\end{enumerate}
Fill in the blanks in each of the Excercises from 34 to 40.

\begin{enumerate}[resume]
\item The vector $\vec{a}+\vec{b}$ bisects the angle between the non-collinear vectors $\vec{a}$ $\text{ and }$ $\vec{b}$ if \rule{1cm}{0.15mm}.


\item If $\vec{r}.\vec{a}=0$, $\vec{r}.\vec{b}=0$ $\text{and}$ $\vec{r}.\vec{c}=0$ for some non-zero vector $\vec{r}$, then the value of $\vec{a}.(\vec{b}$ $\times$ $\vec{c})$ is \rule{1cm}{0.15mm}.


\item The vectors $\vec{a}=3\hat{i}-2\hat{j}+2\hat{k}$ $\text{ and }$ $\vec{b}=\hat{i}-2\hat{k}$ are the adjancent sides of a parallelogram. The acute angle between its diagonals is \rule{1cm}{0.15mm}.


\item The values of $k$ for which $\abs{\vec{ka}}$ $<$ $\abs{\vec{a}}$ $\text{and}$ $k\vec{a}$+$\dfrac{1}{2}$ $\vec{a}$ is parallel to $\vec{a}$ holds true are \rule{1cm}{0.15mm}.


\item The value of the expression $\abs{\vec{a}\times\vec{b}}$+ $({\vec{a}.\vec{b}})$ is \rule{1cm}{0.15mm}.


\item If $\abs{\vec{a}\times\vec{b}}^2$ + $\abs{\vec{a}.\vec{b}}^2$=144 $\text{and}$  $\abs{\vec{a}}$=4, then $\abs{\vec{b}}$ is equal to \rule{1cm}{0.15mm}.


\item If $\vec{a}$ is  any non-zero vector, then $(\vec{a}.\hat{i})\hat{i}$+$(\vec{a}.\hat{j})\hat{j}$+$(\vec{a}.\hat{k})$ $\hat{k}$ equals \rule{1cm}{0.15mm}.

\end{enumerate}

State \textbf{True} or \textbf{False} in each of the following Exercises.

\begin{enumerate}[resume]

\item If $\abs{\vec{a}}$ = $\abs{\vec{b}}$, then necessarily it implies $\vec{a}=\pm\vec{b}$.


\item Position vector of point P is a vector whose intial point is origin.


\item If $\abs{\vec{a}+\vec{b}}$ = $\abs{\vec{a}-\vec{b}}$, then the vectors $\vec{a}$ $\text {and}$ $\vec{b}$ are orthogonal.


\item The formula $(\vec{a}+\vec{b})$ = $\vec{a}$+$\vec{b}$ + 2$\vec{a}\times\vec{b}$ is valid for non-zero vectors $\vec{a}$ $\text{and}$ $\vec{b}$.


\item If $\vec{a}$ $\text{ and }$ $\vec{b}$ are adjacent sides of a rhombus, then $\vec{a}.\vec{b}$.=0.
	\end{enumerate}

\end{document}
