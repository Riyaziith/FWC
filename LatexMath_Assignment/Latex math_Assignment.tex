\documentclass{article} 
\usepackage{amsmath}
\usepackage{graphicx}
\usepackage{amsmath}   
\usepackage{commath}
\usepackage{gensymb}
\usepackage{enumitem}
\usepackage{multicol}
\usepackage[margin=0.5in]{geometry}

\newlist{choices}{enumerate}{1}
\setlist[choices]{label*=(\Alph*)}
\newcommand{\choice}{\item}

\SetEnumitemKey{twocol}{
  before=\raggedcolumns\begin{multicols}{2},
  after=\end{multicols}}

\SetEnumitemKey{threecol}{
  before=\raggedcolumns\begin{multicols}{3},
  after=\end{multicols}}

\SetEnumitemKey{fourcol}{
  before=\raggedcolumns\begin{multicols}{4},
  after=\end{multicols}}
  

\begin{document}
\textbf{}\\
\\Latex Math Assignment ncert-exemplar/12/leep210.pdf-Excercise 10.3\\

\textbf{}\\
\textbf{Short Answer (S.A)}\\
\\
1. Find the unit vector in the direction of sum of vectors $\overrightarrow{a}=2\hat{i}-\hat{j}+\hat{k} \text{ and } \overrightarrow{b}=2\hat{j}+\hat{k}$.
\\
\\
2. If $\overrightarrow{a}=\hat{i}+\hat{j}+2\hat{k} \text{ and }\overrightarrow{b}=2\hat{i}+\hat{j}-2\hat{k}$, find the unit vector in the direction of\\

(i) 6$\overrightarrow{a}$ \hspace{2cm}  (ii) 2$\overrightarrow{a}-\overrightarrow{b}$
\\
\\
3. Find a unit vector in the direction of $\overline{PQ}$, where P and Q have co-ordinates(5,0,8)and (3,3,2),respectively.
\\
\\
4. If $\overrightarrow{a}$ and $\overrightarrow{b}$ are the postion vectors of A and B, respectively, find the position vector of a point C in BA produced such that BC=1.5 BA.
\\
\\
5. Using vectors, find the value of k such that the points (k,-10,3),(1,-1,3)and (3,5,3) are colinear
\\
\\
6. A vector $\overrightarrow{r}$ is inclined at equal angles to the three axes. If the magnitude of $\overrightarrow{r}$ is $2\sqrt{3}$ units, find $\overrightarrow{r}$.
\\
\\
7. A vector $\overrightarrow{r}$ has a magnitude 14 and direction ratios 2,3,-6. Find the direction cosines and components of $\overrightarrow{r}$,given that $\overrightarrow{r}$ makes an acute angle with x-axis.
\\
\\
8. Find a vector of magntude6, which is perpendicular to both the vectors $2\hat{i}-\hat{j}+2\hat{k}$ and $4\hat{i}-\hat{j}+3\hat{k}$ .
\\
\\
9. Find the angle between the vectors $ 2\hat{i}-\hat{j}+\hat{k}$ and $3\hat{i}+4\hat{j}-\hat{k}$.
\\
\\
10. If $\overrightarrow{a}+\overrightarrow{b}+\overrightarrow{c}=0$, show that $\overrightarrow{a}\times\overrightarrow{b}=\overrightarrow{b}\times\overrightarrow{c}=\overrightarrow{c}\times\overrightarrow{a}$. Interpret the result geometrically?
\\
\\
11. Find the sine of the angle between the vectors $\overrightarrow{a}=3\overrightarrow{i}+\overrightarrow{j}+2\overrightarrow{k}$ and $b=2\overrightarrow{i}-2\overrightarrow{j}+4\overrightarrow{k}$.
\\
\\
12. If A,B,C,D  are the points with position vectors $\hat{i}+\hat{j}-\hat{k}, 2\hat{i}-\hat{j}+3\hat{k} , 2\hat{i}-3\hat{k}, 3\hat{i}-2\hat{j}+\hat{k}$, respectively, find the projection of $\overline{AB}$ along $\overline{CD}$.
\\
\\
13. Using vectors, find the area of trianlge ABC with vertices A(1,2,3), B(2,-1,4) and C(4,5,-1).
\\
\\
14. Using vectors, prove that the parallelogram on the same base and between the same parallels are equal in area.
\\
\\
\textbf{Long Answer Questions(L.A)}
\\
\\
15. Prove that in any triangle ABC, cosA=$\frac{b^2+c^2-a^2}{2bc}$, where a,b,c are the magnitudes of the sides opposite to the vertices A,B,C respectively.
\\
\\
16. If $\overrightarrow{a},\overrightarrow{b},\overrightarrow{c}$ determine the vertices of a triangle, show that $\dfrac{1}{2}$ $\left[ \overrightarrow{b}\times\overrightarrow{c}+ \overrightarrow{c} \times\overrightarrow{a}+ \overrightarrow{a} \times\overrightarrow{b} \right]$ gives the vector area of the triangle. Hence deduce the condition that the three points $\overrightarrow{a}$,$\overrightarrow{b}$,$ \overrightarrow{c}$ are collinear. Also find the unit vector of normal to the plane of triangle.
\\
\\
17. Show that area of the parallelogram whose diagonals are given by $\overrightarrow{a}\times\overrightarrow{b}$ is $\dfrac{|a \times\ b|}{2}$. Also find the area of the parallelogram whose diagonals are $2\hat{i}-\hat{j}+\hat{k}$ and $\hat{i}+3\hat{j}-\hat{k}$.
\\
\\
18. If $\overrightarrow{a}=\hat{i}+\hat{j}+\hat{k}$ and $\overrightarrow{b}=\hat{j}-\hat{k}$, find a the vector $\overrightarrow{c}$ such that $\overrightarrow{a}\times\overrightarrow{c}=\overrightarrow{b}$ and $\overrightarrow{a}.\overrightarrow{c}$ = 3.
\\
\\
\\
\textbf{Objective Type Questions}
\\
\\
Choose the correct answer from the given four options in each of the Excercise from 19 to 33(M.C.Q)
\\
\\
19. The vector in the direction of the vector $\hat{i}-2\hat{j}+2\hat{k}$ that has magnitude9 is
\begin {choices}[twocol]
\choice $\hat{i}-2\hat{j}+2\hat{k}$ 		
\choice $\dfrac{\hat{i}-2 \hat{j}+2\hat{k}}{3}$ 
\choice 3$(\hat{i}-2\hat{j}+2\hat{k})$
\choice 9$(\hat{i}-2\hat{j}+2\hat{k})$
\end{choices}
20. The position vector of the point which divides the join of points $2\overrightarrow{a}-3\overrightarrow{b}$ and $\overrightarrow{a}+\overrightarrow{b}$ in the ratio 3:1 is
\begin{choices}[fourcol]
\choice $\dfrac{3\overrightarrow{a}-2\overrightarrow{b}}{2}$
\choice $\dfrac{7\overrightarrow{a}-8\overrightarrow{b}}{4}$
\choice $\dfrac{\overrightarrow{3a}}{4}$	
\choice $\dfrac{\overrightarrow{5a}}{4}$
\end{choices}
21. The vector having intial and terminal points as (2,5,0)and (-3,7,4),respectively is 
\begin{choices}[twocol]
\choice -$\hat{i}+12\hat{j}+4\hat{k}$
\choice -5$\hat{i}+2\hat{j}+4\hat{k}$
\choice 5$\hat{i}+ 2\hat{j}-4\hat{k}$
\choice $\hat{i}+\hat{j}+\hat{k}$
\end{choices}
22. The angles between two vectors $\overrightarrow{a} and \overrightarrow{b}$ with magnitude $\sqrt{3}$ and 4, respectively, and $\overrightarrow{a},\overrightarrow{b}=2\sqrt{3}$ is
\begin{choices}[fourcol]
\choice $\dfrac{\pi}{6}$ 
\choice $\dfrac{\pi}{3}$
\choice $\dfrac{\pi}{2}$
\choice $\dfrac{5\pi}{2}$
\end{choices}
23. Find the value of $\lambda$ such that the vectors $\overrightarrow{a}=2\hat{i}+\lambda\hat{j}+\hat{k}$ and $\overrightarrow{b}=\overrightarrow{i}+2\overrightarrow{j}+3\overrightarrow{k}$ are orthogonal.
\begin{choices}[fourcol]
\choice 0
\choice 1
\choice $\dfrac{3}{2}$
\choice -$\dfrac{5}{2}$
\end{choices}
24. The value of $\lambda$ for which the vectors $3\hat{i}-6\hat{j}+\hat{k}$ and $2\hat{i}-4\hat{j}$+$\lambda\hat{k}$ are parallel is
\begin{choices}[fourcol]
\choice $\dfrac{2}{3}$
\choice $\dfrac{3}{2}$
\choice $\dfrac{5}{2}$
\choice $\dfrac{2}{5}$
\end{choices}
25. The vector from origin to the points A and B are $\overrightarrow{a}=2\hat{i}-3\hat{j}+2\hat{k}$ and $\overrightarrow{b}=2\hat{i}+3\hat{j}+\hat{k}$, respectively, then the area of trianlge OAB is
\begin{choices}[fourcol]
\choice 340
\choice $\sqrt{25}$
\choice $\sqrt{229}$	
\choice $\dfrac{1}{2}\sqrt{229}$
\end{choices}
26. For any vector $\overrightarrow{a}$, the value of $(\overrightarrow{a}\times\overrightarrow{i})^2+(\overrightarrow{a}\times\overrightarrow{j})^2 +(\overrightarrow{a}\times\overrightarrow{k})^2$ is equal to 
\begin{choices}[fourcol]
\choice a
\choice 3a
\choice 4a
\choice 2a
\end{choices}
27. If $|\overrightarrow{a}|=10, |\overrightarrow{b}|=2$ and $\overrightarrow{a},\overrightarrow{b}=12$, then value of $|\overrightarrow{a}\times\overrightarrow{b}|$ is
\begin{choices}[fourcol]
\choice 5
\choice 10
\choice 14
\choice 16
\end{choices}
28. The vectors $\lambda\hat{i}+\hat{j}+2\hat{k},\hspace{1mm} \hat{i}+\lambda\hat{j}-\hat{k}$ and $2\hat{i}-\hat{j}+\lambda\hat{k}$ are coplanar if
\begin{choices}[fourcol]
\choice $\lambda=-2$
\choice $\lambda=0$
\choice $\lambda=1$
\choice $\lambda=1$
\end{choices}
29. If $\hat{a},\hat{b},\hat{c}$ are unit vectors such that $\hat{a}+\hat{b}+\hat{c}=0$, then the value of $\hat{a}.\hat{b}+\hat{b}.\hat{c}+\hat{c}.\hat{a}$ is
\begin{choices}[fourcol]
\choice 1
\choice 3
\choice -$\dfrac{3}{2}$
\choice None of these
\end{choices}
30. Projection vector of $\overrightarrow{a}$ on $\overrightarrow{b}$ is
\begin{choices}[fourcol]
\choice $\left(\dfrac{\overrightarrow{a}.\overrightarrow{b}}{|\overrightarrow{b}|^2}\right)$
\choice $\dfrac{\overrightarrow{a}.\overrightarrow{b}}{|\overrightarrow{b}|}$
\choice $\dfrac{\overrightarrow{a}.\overrightarrow{b}}{|\overrightarrow{a}|}$
\choice $\left(\dfrac{\overrightarrow{a}.\overrightarrow{b}}{|\overrightarrow{a}|^2}\right)$
\end{choices}
31. $\overrightarrow{a}$,$\overrightarrow{b}$,$\overrightarrow{c}$ are the three vectors such that $\overrightarrow{a}$+$\overrightarrow{b}$+$\overrightarrow{c}=0$ and $|\overrightarrow{a}|=2, |\overrightarrow{b}|=3, |\overrightarrow{c}|=5$, the value of $\overrightarrow{a}.\overrightarrow{b}+\overrightarrow{b}.\overrightarrow{c}+\overrightarrow{c}.\overrightarrow{a}$ is
\begin{choices}[fourcol]
\choice 0
\choice 1
\choice -19
\choice 38
\end{choices}
32. If $|\overrightarrow{a}|=4$ and $-3\leq\lambda\leq2,$ then the range of $|\lambda\overrightarrow{a}|$ is
\begin{choices}[fourcol]
\choice $\left[0,8\right]$
\choice $\left[-12,8\right]$
\choice $\left[0,12\right]$
\choice $\left[8,12\right]$
\end{choices}
33. The number of vectors of unit length perpendicular to the vectors $\overrightarrow{a}=2\overrightarrow{i}+\overrightarrow{j}+2\overrightarrow{k}$ and $\overrightarrow{b}=\overrightarrow{j}+\overrightarrow{k}$ is
\begin{choices}[fourcol]
\choice one
\choice two
\choice three
\choice infinite
\end{choices}
Fill in the blanks in each of Excersices from 34 to 40.
\\
\\
34. The vector $\overrightarrow{a}+\overrightarrow{b}$ bisects the angle between the non-collinear vectors $\overrightarrow{a}$ and $\overrightarrow{b}$ if \rule{1cm}{0.15mm}
\\
\\
35. If $\overrightarrow{r}.\overrightarrow{a}=0, \overrightarrow{r}.\overrightarrow{b}=0$ and $\overrightarrow{r}.\overrightarrow{c}=0$ for some non-zero vector $\overrightarrow{r},$ then the value of $\overrightarrow{a}.(\overrightarrow{b}\times\overrightarrow{c})$ is \rule{1cm}{0.15mm}
\\
\\
36. The vectors $\overrightarrow{a}=3\hat{i}-2\hat{j}+2\hat{k}$ and $\overrightarrow{b}=-\hat{i}-2\hat{k}$ are the adjancent sides of a parallelogram. The acute angle between its diagonals is \rule{1cm}{0.15mm}
\\
\\
37. The values of k for which $|k\overrightarrow{a}|$ $<$ $|\overrightarrow{a}|$and k$\overrightarrow{a}+\dfrac{1}{2}\overrightarrow{a}$ is parallel to $\overrightarrow{a}$ holds true are \rule{1cm}{0.15mm}
\\
\\
38. The value of the expression $|\overrightarrow{a}\times\overrightarrow{b}|^2$ +$ (\overrightarrow{a}.\overrightarrow{b})^2$ is \rule{1cm}{0.15mm}
\\
\\
39. If $|\overrightarrow{a}\times\overrightarrow{b}|$+$|\overrightarrow{a}.\overrightarrow{b}|$=144 and $|\overrightarrow{a}|=4$, then $|\overrightarrow{b}|$ is equal to \rule{1cm}{0.15mm}
\\
\\
40. If $\overrightarrow{a}$ is  any non-zero vector, then $(\overrightarrow{a}.\hat{i})$$\hat{i}$+$(\overrightarrow{a}$.$\hat{j})\hat{j}$+ $(\overrightarrow{a}.\hat{k})$$\hat{k}$ equals \rule{1cm}{0.15mm}
\\
\\
State \textbf{True} or \textbf{False} in each of the Excercises.
\\
\\
41. If $|\overrightarrow{a}|=|\overrightarrow{b}|$, then necessarily it implies $\overrightarrow{a}=\pm\overrightarrow{b}.$
\\
\\
42. Position vector of point P is a vector whose intial point is origin.
\\
\\
43. If $|\overrightarrow{a}+\overrightarrow{b}|$=$|\overrightarrow{a}-\overrightarrow{b}|$, then the vectors $\overrightarrow{a}$ and $\overrightarrow{b}$ are orthogonal.
\\
\\
44. The formula $(\overrightarrow{a}+\overrightarrow{b})$= $\overrightarrow{a}^2+\overrightarrow{b}^2+2\overrightarrow{a}\times\overrightarrow{b}$ is valid for non-zero vectors $\overrightarrow{a}$ and $\overrightarrow{b}.$
\\
\\
45. If $\overrightarrow{a}$ and $\overrightarrow{b}$ are adjacent sides of a rhombus, then $\overrightarrow{a}.\overrightarrow{b}.$=0.
\\
\\
\end{document}
